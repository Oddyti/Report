\documentclass{../source/Experiment}

\major{信息工程}
\name{姚桂涛}
\title{波导传输线负载特性测量与阻抗匹配}
\stuid{3190105597}
\college{信息与电子工程学院}
\date{\today}
\lab{东4-221}
\course{电磁场与电磁波}
\instructor{王子立}
\grades{}
\expname{波导传输线负载特性测量与阻抗匹配}
\exptype{}
\partner{华天择}

\usepackage{caption}

\DeclareCaptionLabelSeparator{twospace}{\, }
\captionsetup{labelsep = twospace}

\begin{document}
    \makecover
    \makeheader

    \section{实验目的}
    了解波导传输线的基本特性,容性膜片的负载特性及阻抗匹配方法。
    
    覆盖的基本概念:
    \begin{itemize}
        \item 波导的传输线模型
        \item 波导色散特性——波导波长
        \item 阻抗及匹配
        \item Smith圆图
    \end{itemize}

    \section{实验过程及结果}

    \section{实验结果分析}
        \subsection{计算波导波长$\lambda$}
        \subsection{计算$TE_{10}$模的波导波长$\lambda_E$,并比较}
        \subsection{计算$\rho $,读出$\Gamma $和归一化阻抗值。}
        \subsection{计算用单销钉调节匹配后的驻波系数。}
        \subsection{,计算匹配状态时销钉所呈现的归一化电抗值。}

        \subsection{回答问题:}
            \subsubsection{}
            \subsubsection{}
            \subsubsection{}
            \subsubsection{}
            \subsubsection{}

    \section{实验总结与心得体会}
    $h[n]=\frac{\sin \left[w_{0}\left(n-n_{0}\right)\right]}{\pi\left(n-n_{0}\right)}$
\end{document}