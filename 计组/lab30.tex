\documentclass{../source/Experiment}

\major{信息工程}
\name{姚桂涛}
\title{基于RV32I指令集的RISC-V微处理器设计}
\stuid{3190105597}
\college{信息与电子工程学院}
\date{\today}
\lab{教11-400}
\course{计算机组成与设计}
\instructor{屈民军、唐奕}
\grades{}
\expname{RISC-V微处理器设计}
\exptype{设计}
\partner{}
\begin{document}
    \makecover
    \makeheader
    \section{实验目的}
    
    \section{实验任务}
        \subsection{基本要求}
        设计一个流水线RISC-V微处理器,具体要求如下所述。

        (1) \, 至少运行下列RV32I核心指令。

        算术运算指令:add、sub、addi

        逻辑运算指令:and、or、xor、slt、sltu、di、ori、xori、slti、sltiu

        移位指令:sll、srl、sra、slli、srli、srai

        条件分支指令:beq、bne、blt、bge、bltu、eu

        无条件跳转指令:jal、jalr

        数据传送指令:lw、sw、lui、auipc
    
        空指令:nop

        (2) \, 采用 5 级流水线技术,对数据冒险实现转发或阻塞功能。

        (3) \, 在 Nexys Video 开发系统中实现RISC-V 微处理器, 要求 CPU 的运行速度大于 25MHz。
        \subsection{扩展要求}
        (1) \, 要求设计的微处理器还能运行lb、lh、ld、lbu、lhu、lwu、sb、sh 或 sd 等字节、半字和双字数据传送指令。
        
        (2) \, 要求设计的CPU 增加异常(exception)、自陷(trap)、中断(interrupt)等处理方案。
        




\end{document}