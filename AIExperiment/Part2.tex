\documentclass{../source/Experiment}

\major{信息工程}
\name{姚桂涛}
\title{回归模型}
\stuid{3190105597}
\college{信息与电子工程学院}
\date{\today}
\lab{教11-400}
\course{人工智能实验}
\instructor{胡浩基、魏准}
\grades{}
\expname{回归模型}
\exptype{设计验证}
\partner{}
\begin{document}
    \makecover
    \section{实验题目}
        \subsection{线性回归}
        编写程序,通过产生的附加噪声的随机数据,(1)做线性回归(即求出𝜽);(2)利用取得的回归模型,对x=0和x=2两个点做预测;(3)对上述随机数据和预测结果利用plot可视化。

        \subsection{多项式回归}
       
        通过网上调研了解Scikit -Learn中的LinearRegression模块,编写程序,利用LinearRegression模块:(1)对上述产生的二级多项式数据(X\_poly, y) 做线性回归。(2) 可视化回归模型和数据。


        \subsection{逻辑回归}

        (1)查阅网上资料,通过sklearn.linear\_model中的LogisticRegression,利用花瓣宽度,实现一个分类器检测维吉亚鸢尾花。
        (2)可视化花瓣宽度0-3 cm之间的鸢尾花模型估算出的概率
    
    \section{实验代码}
    \subsection{线性回归}
    \lstinputlisting[
        language  =   Python,
        title = {Linear Regression}
        ]{./Part2/lab1.py}
    \lstinputlisting[
            language  =   Python,
            title = {Polynomial Regression}
        ]{./Part1/lab2.py}
    \subsection{多项式回归}
    \lstinputlisting[
            language  =   Python,
            title = {Logistic Regression}
        ]{./Part2/lab3.py}
    \subsection{多项式回归}

    \section{实验结果}
    

\end{document}


