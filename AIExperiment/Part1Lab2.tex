\documentclass{../source/Experiment}

\major{信息工程}
\name{姚桂涛}
\title{k-近邻算法}
\stuid{3190105597}
\college{信息与电子工程学院}
\date{\today}
\lab{教11-400}
\course{人工智能实验}
\instructor{胡浩基、魏准}
\grades{}
\expname{k-近邻算法}
\exptype{设计验证}
\partner{}
\begin{document}
    \makecover
    \section{实验题目}
        \subsection{kNN代码实现-AB分类}
        采用测量不同特征值之间的距离方法进行分类,用所给的函数创建具有两个特征与一个标签类型的数据作为训练集,编写classify0函数对所给的数据进行AB分类。
        \subsection{k-近邻算法改进约会网站的配对效果}
    k-近邻算法改进约会网站的配对效果

    通过收集的一些约会网站的数据信息,对匹配对象的归类:不喜欢的人、魅力一般的人、极具魅力的人。

    数据中包含了3种特征:

    每年获得的飞行常客里程数、玩视频游戏所耗时间百分比、每周消费的冰淇淋公升数
    
    \section{实验代码}
    \subsection{kNN代码实现-AB分类}
    \lstinputlisting[
        language  =   Python,
        title = {kNN}
]{./Part1/kNN.py}
\lstinputlisting[
        language  =   Python,
        title = {tests}
]{./Part1/tests.py}
    \subsection{k-近邻算法改进约会网站的配对效果}
    \lstinputlisting[
        language  =   Python
]{./Part1/lab2.py}
    \section{实验结果及分析}
    \subsection{kNN代码实现-AB分类}
    \begin{lstlisting}[language=Python]
    分类结果
    [0, 0] B
    [0.8, 0.7] A
    \end{lstlisting}
    \subsection{k-近邻算法改进约会网站的配对效果}
    \begin{lstlisting}[language=Python]
    前20个数据测试结果和原数据比较
    ---------------------——---
        TestResult  OriginTest
    0            2           2
    1            3           3
    2            1           3
    3            2           2
    4            2           2
    5            3           3
    6            3           3
    7            2           2
    8            1           1
    9            1           1
    10           1           1
    11           3           3
    12           2           2
    13           2           2
    14           1           1
    15           2           2
    16           1           1
    17           2           2
    18           1           1
    19           3           3
    ---------------------——---
    正确率97.00%
    \end{lstlisting}

    从实验结果可以看出,通过k-近邻算法改进后的约会网站的配对效果比较显著,多次随机划分测试集和训练集后发现正确率基本可以达到90\%以上。


\end{document}


