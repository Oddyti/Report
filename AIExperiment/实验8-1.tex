\documentclass{../source/Experiment}

\major{信息工程}
\name{}
\title{自编码器(Autoencoder)}
\stuid{}
\college{信息与电子工程学院}
\date{\today}
\lab{教11-400}
\course{人工智能实验}
\instructor{胡浩基、魏准}
\grades{}
\expname{自编码器(Autoencoder)}
\exptype{设计验证}
\partner{}
\begin{document}
\makecover
\section{实验题目}
\subsection{实验8-1}
使用Keras工具包编写自动编码器进行图像去噪:把训练样本用噪声污染,利用Keras工具包编写一个卷积神经网络自编码器,使解码器解码出干净的照片(训练、测试数据数目根据自己电脑内存取值):

(1)  搭建卷积神经网络自编码器,尽量取得低的相对误差;

(2)  固定loss 为MSE,比较至少3种不同优化器的训练测试结果;

(3)  固定优化器为Adam, 比较至少3种不同loss的训练测试结果;

(4)  找出相对误差小于50\%时能够获得的最大压缩比。

\section{实验结果}
\subsection{实验8-1}

固定loss 为MSE:

\begin{lstlisting}[language = python]
        optimizer='SGD', loss = 'MSE'
        loss: 0.0363 - val_loss: 0.0352

        optimizer='adam', loss = 'MSE'
        loss: 0.1120 - val_loss: 0.1140
            
        optimizer='Adagrad', loss = 'MSE'
        loss: 0.1120 - val_loss: 0.1140
        \end{lstlisting}

固定优化器为Adam:

\begin{lstlisting}[language = python]
        optimizer='adam', loss = 'MSE'
        loss: 0.1120 - val_loss: 0.1140

        optimizer='adam', loss = 'binary_crossentropy'
        loss: 0.0986 - val_loss: 0.0976

        optimizer='adam', loss = 'MAE'
        loss: 0.1307 - val_loss: 0.1325
        \end{lstlisting}

找出相对误差小于50\%时能够获得的最大压缩比:
\begin{lstlisting}[language = python]
        相对误差:0.41
        压缩比:0.0625
        \end{lstlisting}

\end{document}


