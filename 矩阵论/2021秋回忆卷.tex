\documentclass{../source/Paper}

%关键信息输入
\major{}
\name{}
\articletitle{}
\stuid{}
\college{}
\date{\today}
\course{}
\instructor{}
%摘要
\Abstract{}
%关键词
\Keyword{}

\pagestyle{plain}

\begin{document}
    \begin{center}
        \bfseries\huge{2021秋矩阵论回忆卷}


        \large{by\,Oddyti}
    \end{center}
    \section{}
        \bfseries{1.}

        (1) 证明:$tr(\boldsymbol{AB}) = tr(\boldsymbol{BA})$
        
        (2) 证明:若$\boldsymbol P^{-1} \boldsymbol A \boldsymbol P =  \boldsymbol B$,则$tr(\boldsymbol{A}) = tr(\boldsymbol{B}) = \sum \nolimits_{i = 1}^n \lambda_i$

        \bfseries{2.}
        若$\boldsymbol A^T = - \boldsymbol A$,证明$e^{\boldsymbol A}$为酉矩阵。
    \section{}
        $\boldsymbol A = \boldsymbol U \begin{bmatrix}
            \boldsymbol {\sum} & \boldsymbol 0 \\ \boldsymbol 0 &  \boldsymbol 0
        \end{bmatrix}V^H$,若$\boldsymbol a = \boldsymbol V \begin{bmatrix}
            \boldsymbol {\sum}^{-1} & \boldsymbol 0 \\ \boldsymbol 0 &  \boldsymbol 0
        \end{bmatrix}\boldsymbol U^H\boldsymbol b$,求证$||\boldsymbol A \boldsymbol a - \boldsymbol b||_2 \leq ||\boldsymbol A \boldsymbol x - \boldsymbol b||_2 $
    \section{}
        $$min f({\boldsymbol x}) \quad s.t. \quad g_i(\boldsymbol x) \leq 0 \, i = 0,1,2,...I ,\, \quad h_j(\boldsymbol x) = 0 \, j = 0,1,2,...J$$

        \bfseries{1.} 混合外罚函数进行无约束,一个。

        \bfseries{2.} 混合内罚函数进行无约束,两个。

        \bfseries{3.} 混合拉格朗日函数进行无约束,一个。

    \section{}
        $\boldsymbol A$、$\boldsymbol B$为Hermitian矩阵,$\boldsymbol x \neq 0$。 

        \bfseries{1.} Rayleigh商为$R(x) =\displaystyle  \frac{\boldsymbol x ^H \boldsymbol A \boldsymbol x}{\boldsymbol x^H \boldsymbol x}$,求$R(x)$的最大值以及对应的$\boldsymbol x$。

        \bfseries{2.} 广义Rayleigh商为$R(x) = \displaystyle  \frac{\boldsymbol x ^H \boldsymbol A \boldsymbol x}{\boldsymbol x^H \boldsymbol B \boldsymbol x}$,求$R(x)$的最大值以及对应的$\boldsymbol x$。
        
    \section{}
        \bfseries{1.} 证明:$d[tr(\boldsymbol X^T\boldsymbol X)] = 2tr(\boldsymbol X^Td\boldsymbol X)$

        \bfseries{2.} 求$f(\boldsymbol x) = \boldsymbol a^T \boldsymbol x$与$f(\boldsymbol x) = \boldsymbol x^T \boldsymbol A \boldsymbol x$的Hessian矩阵。
    
    \section{}
        $\boldsymbol A  = \begin{bmatrix}
            -2 & 1 & 0 \\
            -4 & 2 & 0 \\
            1 & 0 & 1 
        \end{bmatrix}$

        \bfseries{1.} 求$\boldsymbol A  $的特征多项式

        \bfseries{1.} 求$\boldsymbol {sinA}  $
        
        提示:$\boldsymbol {sinA}   = \sum _{n = 0}^{\infty} \displaystyle  \frac{(-1)^n \boldsymbol A^(2n-1)  }{(2n+1)!}$

    \section{} 
        \begin{center}
            已知:$\boldsymbol y = \boldsymbol X \boldsymbol \beta + \boldsymbol \varepsilon ,\, E\{\boldsymbol \varepsilon \} = 0,\,E\{\boldsymbol \varepsilon ^H \boldsymbol \varepsilon \} = \sigma^2 \boldsymbol I$
        \end{center}

        现在设计:$\boldsymbol e = \boldsymbol A\boldsymbol y$满足$E\{\boldsymbol e - \boldsymbol \varepsilon\} = 0$使得$E\{(\boldsymbol e-\boldsymbol \varepsilon)^H(\boldsymbol e-\boldsymbol \varepsilon)\}$最小,请证明上述优化问题可以等价为:
        $$min[tr(\boldsymbol A^T\boldsymbol A)-2tr(\boldsymbol A)] ,\quad s.t.\, \boldsymbol A\boldsymbol X = \boldsymbol O$$

    \section{简答题} 

    \bfseries{1.} 说明条件数的物理意义以及与奇异值的关系。

    \bfseries{1.} 说明标准正交变换的过程以及与白(色)噪声的关系。

    \bfseries{1.} 说明Tikhonov正则化与反正则化的目的。

\end{document}